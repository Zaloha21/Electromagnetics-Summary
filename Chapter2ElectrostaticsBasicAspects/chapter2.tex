\chapter{Electrostatics: Basic Aspects}
\section{Coulombs law and Electric fields}
The electrical field is derived from coulombs law, in this case between charge 1 and 2.
\begin{equation}
    \textbf{F}_{12} = k\frac{q_1q_2}{r^3}\textbf{r}
\end{equation}
Where $k = \frac{1}{4\pi \epsilon_0}$ in (m/F). The electric field strength is dependent on location and a more generalised representation of the force between the charges. The equation of electric field strength \textbf{E(\textbf{r})} is as follows.
\begin{equation}
    \textbf{E\textbf{(r)}} = k \frac{q_1(\textbf{r}-\textbf{r}_1)}{|\textbf{r}-\textbf{r}_1|^3}
\end{equation}
Or in the case of a charge volume instead of a point charge the following holds:
\begin{equation}
    \textbf{E(\textbf{r})} = k \int_V \rho(\textbf{r}')\frac{(\textbf{r}-\textbf{r}')}{|\textbf{r}-\textbf{r}'|^3}d\textbf{r}'
\end{equation}
Where \textbf{r}' is the location where an observation takes place. (The cubed factor comes from the fact that an unit vector can be expressed in a fraction, multiplied with the integrand $\displaystyle \textbf{r} = \frac{r_1-r_2}{|r_1-r_2|}$. Electric field density \textbf{D} can be described as $\textbf{D} = \epsilon \textbf{E}$ or $\textbf{D} = \frac{E}{k}$
\section{Electrostatic Potential}
The electric field intensity can also be derived from the electrostatic potential as demonstrated below.
\begin{equation}
    \begin{aligned}
        \nabla_r \frac{1}{|\textbf{r}-\textbf{r'}|} = - \frac{(\textbf{r}-\textbf{r}')}{|\textbf{r}-\textbf{r}'|^3} \text{   as   } |\textbf{r}| = \sqrt{x^2+y^2+z^2}\\
        \textbf{E(\textbf{r})} = -k \int_V \rho  (\textbf{r}')\nabla_r \frac{1}{|\textbf{r}-\textbf{r}'|^3}d\textbf{r}' = -\nabla\phi
    \end{aligned}
\end{equation}
This implies that the electric field is the gradient of the electrostatic potential $\phi(\textbf{r})$. And since $\nabla \times \nabla = 0$;
\begin{equation}
    \nabla \times \textbf{E} = 0
\end{equation}
This tells us that the rotation of the electric field is 0; the electric field is curl-free/rotation free. Next the flux is considered. The flux is $\textbf{E} \cdot d\textbf{F}$ and represents the amount of vector field coming through a surface. So the equation for flux is as follows.
\begin{equation}
    \Phi(\textbf{r}) = k\int_{V(F)} \rho(\textbf{r}') d\textbf{r}'\int_F\frac{(\textbf{r}-\textbf{r}_1)}{|\textbf{r}-\textbf{r}_1|^3} d\textbf{F(r)}
\end{equation}
Where V is the volume enclosed by surface F. Assuming that \textbf{r}' is outside of V means that the following holds. (Divergence theorem) 
\begin{equation}
    \nabla \cdot \textbf{E} = 4k\pi \rho
\end{equation}
For a conductive shell with no enclosed charge no flux is apparent. This is Gauss Law, which says charge is the source of electric fields.
\section{The Equations of Electrostatics}

The three important equations of electrostatics are:
\begin{equation}
    \begin{aligned}
        &\textbf{E} = -\nabla \phi \\
        &\nabla \cdot \textbf{E} = 4\pi k \rho(\textbf{r}) \text{   or   } \int\textbf{E}\cdot d \textbf{F} = 4\pi k \Sigma_i q_i\\
        &\nabla \times \textbf{E} = 0
    \end{aligned}
\end{equation}

\noindent From the first two equations the Poisson equation can be determined.
\begin{equation}
\begin{aligned}
   &\Delta \phi = -4\pi k \rho(\textbf{r}) = \nabla \cdot \nabla \phi\\
\end{aligned}
\end{equation}
\noindent Or in Cartesian (x,y,z) and cylindrical coordinates(r,$\theta$,z) respectively:

\begin{equation}
    \begin{aligned}
    &\Delta \phi = \frac{\partial^2 \phi}{\partial x^2} + \frac{\partial^2 \phi}{\partial y^2} + \frac{\partial^2 \phi}{\partial z^2}\\
    &\Delta \phi= \frac{1}{r}\frac{\partial}{\partial r}\left(r\frac{\partial \phi}{\partial r}\right) + \frac{1}{r}\frac{\partial^2 \phi}{\partial \theta^2} + \frac{\partial^2 \phi}{\partial z^2}
    \end{aligned}
\end{equation}

\noindent With these it can be concluded that the potential $\phi$ outside of a uniformly charged sphere where $ \displaystyle \rho(\textbf{r}) = 0$ is $\displaystyle \phi = - \frac{constant}{r}$


\section{Dirac's Delta Distribution}
The delta function is an approximate impulse. It is called a function but in reality it is a distribution, defined as a functional by the following property
\begin{equation}
    f(a) =  \int_{-\infty}^\infty f(x)\delta(x-a)dx
\end{equation}
The quantity $\delta(x)$ can be represented as
\begin{equation}
    \delta (x) = \lim_{x\to 0} \frac{1}{2\sqrt{\pi a}}e^{\frac{x^2}{4a}}
\end{equation}
Which entails that $\delta(x)$ is infinity for x = 0 and 0 otherwise. The following are key properties of the delta function.
\begin{itemize}
    \item $\delta(x) = \delta(-x)$
    \item $\delta'(x) = - \delta'(-x)$
    \item $x\delta(x) = 0$
    \item $x\delta'(x) = -\delta(x)$
    \item $\delta(ax) = \frac{1}{a} \delta(x)$
\end{itemize} 
\noindent When it holds that $\delta(\textbf{r}) = \delta(x)\delta(y)\delta(z)$ a solution to $\Delta G(\textbf{r}) = \delta(\textbf{r}) $ is called Green's Function. When $G(\textbf{r})$ is known, the inhomogenious solution of the Poisson equation $\phi(\textbf{r})$ of $\Delta\phi = -4k\pi \rho$ follows from the relation as shown here:

\begin{equation}
    \phi(r) = -k \int G(\textbf{r}-\textbf{r}')4\pi \rho d \textbf{r}'
    \label{2.5 Dirac relation}
\end{equation}

\noindent This holds as the Laplacian ($\Delta = \nabla^2$) of this relation yields:

\begin{equation}
    \Delta \phi(\textbf{r}) = -k\int \delta(\textbf{r}-\textbf{r}') 4\pi \rho(\textbf{r}') d\textbf{r}' = -4k\pi\rho(\textbf{r})
\end{equation}

\noindent For $\displaystyle \phi(\textbf{r}) = k \int \frac{\rho(\textbf{r}')d\textbf{r}'}{|\textbf{r}-\textbf{r}'|}$ to satisfy equation \ref{2.5 Dirac relation} the following must hold:

\begin{equation}
    \begin{aligned}
    &G(\textbf{r}-\textbf{r}') = -\frac{1}{4\pi |\textbf{r}-\textbf{r}'|}\\
    &\text{Or in other words}\\
    &\Delta_{\textbf{r}} = -4\pi \delta(\textbf{r})
    \end{aligned}
    \label{2.5poisson}
\end{equation}
Which makes sense for $\textbf{r} \neq 0$ which after the following:

\begin{equation}
\begin{array}{c}
    \displaystyle
r=\sqrt{x^{2}+y^{2}+z^{2}}, \quad \triangle=\frac{\partial^{2}}{\partial x^{2}}+\frac{\partial^{2}}{\partial y^{2}}+\frac{\partial^{2}}{\partial z^{2}} \\

\frac{\partial}{\partial x}\left(\frac{1}{r}\right)=-\frac{x}{r^{3}}, \quad \frac{\partial^{2}}{\partial x^{2}}\left(\frac{1}{r}\right)=-\frac{1}{r^{3}}+\frac{3 x^{2}}{r^{5}}\\

\triangle \frac{1}{r}=-\frac{3}{r^{3}}+\frac{3 r^{2}}{r^{5}}=0
\end{array}
\end{equation}
Can be integrated in the following relation:
\begin{equation}
    \begin{aligned}
    &\int_V \Delta_\textbf{r} \frac{1}{\textbf{r}}d \textbf{r} = \int_V \nabla \cdot (\nabla\frac{1}{r})d\textbf{r} = \int_F (\nabla \frac{1}{r}) \cdot d\textbf{F}\\
   & = -\int_F \frac{\textbf{r}}{r^3}\cdot d \textbf{F} = -\int \frac{dF}{r^2}= -\int d\Omega = -4\pi\\
   \end{aligned} 
\end{equation}

\noindent This then verifies \ref{2.5poisson} and then shows that the Green's function is the potential of a negative unit point charge multiplied with $\epsilon_0$.
\begin{equation}
    \rho(\textbf{r}') = -\epsilon_0 \delta(\textbf{r}')
\end{equation}
When this is kept in mind the equation for the electrostatic potential is:
\begin{equation}
    \phi(\textbf{r})= -k \int G(\textbf{r}-\textbf{r}')4 \pi \rho(\textbf{r}') d \textbf{r}' = G(\textbf{r})
\end{equation}
\section{Potential Energy of Charges}
When a charge q is moved from A to B against the field \textbf{E} the work done can be described as follows:
\begin{equation}
    \begin{aligned}
        W &= - \int_A^B q\textbf{E(r)} \cdot d\textbf{r} = q\int_A^B \nabla\phi \cdot d\textbf{r} =  q\int_A^B d\phi \\    
          &= q(\phi(B)-\phi(A))\\
    \end{aligned}
\end{equation}

\noindent The minus sign infront of the integral represents the opposing force against the electricfield \textbf{E}. Consequently, as the path from A to B does not matter the field
is path-independent. As such $q\textbf{E}$ is a conservative field which implies again that $\nabla \times \textbf{E} = 0$.

\section{The Electric Field at Charged Surfaces}
Allready well known. It contains the following:

\begin{itemize}
    \item Charge per unit area ($\sigma (\textbf{r})$).
    \item Field duality on both sides of a charged plate ($2E = 4\pi k \sigma or E = \frac{\sigma}{2 \epsilon_0}$ for both sides)
    \begin{itemize}
        \item Which comes from $(\mathbf{E_{2}} - \mathbf{E_{1}}) \cdot \mathbf{n} = 4 k \pi \sigma (\mathbf{r}) $
        \item This holds for an infinitesimally thin plate
    \end{itemize}
\end{itemize}

\section{Examples}
This section contains the names of the examples (regarding condensers) used in the book:
\begin{itemize}
    \item The Parallel Plate condenser or capacitor (also as introduction to the dipole)
    \item The spherical condenser (Charge on inner side and outter side of shell, determination of capacitance)
    \item The van den Graaff generator (Calculating when sparks occur)
    \item The cylindrical condenser (Finding C of a cylinder of $r_1,r_2$, no negligeable thickness with charges in and outside)
    \item The dipole (Finding the dipole moment and potential of it in space)
\end{itemize}

\section{Conductiors and Electrical Screening}
For a conductor inside of which charges can move freely the field inside vanishes. 
This means that if a conductor were to be given a charge, the charge is only on the outter surface of the conductor. The surface is an equipotential surface. (this way faraday cages are made)\\
\\
\noindent It can thus be said that for a conductor with surface F:

\begin{equation}
    \int_F \textbf{E} \cdot d\textbf{F} = 0
\end{equation}

A conductor is earthed if the potential is the same as at $\infty$. The potential equation, obtained from the poisson equation, can be used with the superposition principle to relate charge with potential.
\begin{equation}
    \quad q_i = \sum_{j=1}^nC_{ij}\phi_i
\end{equation}
Electrical Screening is when one conductor is surrounded by another. This means that a charge enclosed by 2 other conductors is not influenced by outside fields. This happens when the middle conductor
is earthed (grounded).\\
\\
\noindent The section then showcases multiple examples of screening and charges on conductor surfaces. These are as follows:
\begin{itemize}
    \item Two hemispherical shells (a charged shell is split in 2, find the force to keep it together)
    \item Two connected shells (2 charged shells are connected, find how much the energy has lowered on a shell)
    \item The charges of connected spherical shells (find charge density on surface after connection)
\end{itemize} 

\section{Energy of charge distributions}
The potential of a point charge $q_i$ at the point $\textbf{r}_i$ in the field $\scriptstyle \textbf{E} = - \nabla \phi$
of a point charge $q_k$ at point $\textbf{r}_k$ is the coulomb potential energy. Which is the following:
\begin{equation}
    k \frac{q_iq_k}{|\textbf{r}_i-\textbf{r}_k|}, (i \neq k)
\end{equation}
The potential of N charges $q_i \quad (1,2,3,...,i)$ and $q_k \quad (1,2,3,...,k)$ is correspondingly
\begin{equation}
    W = \frac{1}{2}\sum_i\sum_{k\neq i}\frac{q_iq_k}{|\textbf{r}_i-\textbf{r}_k|}
\end{equation}
In the case of a continuous charge distribution $\rho(\textbf{r})$ the following relation holds:

\begin{equation}
    \begin{aligned}
        &W =  \frac{1}{2}k \int dV\int dV'\frac{\rho(\textbf{r})\rho(\textbf{r}')}{|\textbf{r}-\textbf{r}'|} \\
        &\quad = \frac{1}{2}\int dV \rho(\textbf{r})\phi(\textbf{r}) \text{, Joules} \\
        &\phi(\textbf{r}) = k\int  dV'\frac{\rho(\textbf{r})\rho(\textbf{r}')}{|\textbf{r}-\textbf{r}'|} \text{, Volts}\\ 
    \end{aligned}
\end{equation}

In the case of surface charges the following can be used for a surface F:
\begin{equation}
    W=\frac{1}{2} k \int_{F} d F(\mathbf{r}) \int_{F} d F\left(\mathbf{r}^{\prime}\right) \frac{\sigma(\mathbf{r}) \sigma\left(\mathbf{r}^{\prime}\right)}{\left|\mathbf{r}-\mathbf{r}^{\prime}\right|}
\end{equation}
For when F consists out of 2 condenser plates with area $F_1$ and $F_2$ W can also be expressed in terms of \textbf{E}.\\
\\
\noindent Recall that
\begin{equation}
     \nabla^2 \phi = \triangle\phi = -4k\rho(\textbf{r})
\end{equation}
with equation (2.25) becomes
\begin{equation}
    W = - \frac{1}{8k} \int \phi \nabla^2 \phi dV
\end{equation}
However, it holds that
\begin{equation}
    \nabla \cdot (\phi \nabla \phi) = \phi \triangle \phi + (\nabla\phi)^2
\end{equation}
so that
\begin{equation}
    W=-\frac{1}{8 \pi k} \int d V\left[\nabla \cdot\{\phi(\nabla \phi)\}-(\nabla \phi)^{2}\right]
\end{equation}
which by Gauss' divergence theorem turns into 
\begin{equation}
    \int dV \nabla \cdot (\phi \nabla \phi) = \int d\textbf{F} \cdot (\phi \nabla \phi)
\end{equation}

\noindent Now it is known that $\phi \sim 1\frac{1}{r},\nabla\phi \sim \frac{1}{r^2}$. When integrating over an infinitely large surface (i.e. where $r \rightarrow \infty$)
the integral vanishes, as $dF \propto r^2 d\Omega$. So when $r \rightarrow \infty$ it follows that (as $\textbf{E} = -\nabla \phi$):

\begin{equation}
    W = \frac{1}{8\pi k} \int_{V_\infty} dV \textbf{E}^2 
\end{equation}
Or as earlier learnt $\displaystyle W = \frac{\textbf{E}^2}{8\pi k} = \frac{1}{2}\epsilon_0 \textbf{E}^2 $.\\
\\
Again more examples are given:
\begin{itemize}
    \item The gravitational Potential
    \begin{itemize}
        \item Finding the needed mass density equivalent to the energy density of the field of a charge to find the mass density of a charge. 
    \end{itemize}
    \item Energy of the spherical condenser
    \begin{itemize}
        \item Find the electrostatic potential/energy of a spherical condenser.
    \end{itemize}
    \item A charged particle inside a charge density
    \item A multivalued potential
    \begin{itemize}
        \item Finding the equipotential surfaces of the potential $\phi = tan^{-1}(\frac{x}{y})$
    \end{itemize}
\end{itemize}
